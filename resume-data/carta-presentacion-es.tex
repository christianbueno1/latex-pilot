%-------------------------
% Carta de Presentación
% Autor : Christian Bueno
% Basado en: https://github.com/christianbueno1/latex-pilot
% Licencia : MIT
%-------------------------

\documentclass[a4paper,11pt]{article}

% Paquetes esenciales
\usepackage[empty]{fullpage}
\usepackage[hidelinks]{hyperref}
\usepackage[normalem]{ulem}
\usepackage{xcolor}
\usepackage[spanish,es-noquoting]{babel}
\usepackage[default]{sourcesanspro}

\urlstyle{same}

% Configuración de márgenes
\pagestyle{empty}
\setlength{\tabcolsep}{0in}

\addtolength{\oddsidemargin}{-0.5in}
\addtolength{\topmargin}{-0.5in}
\addtolength{\textwidth}{1.0in}
\addtolength{\textheight}{1.0in}

\raggedbottom{}
\raggedright{}

% Comandos personalizados
\newcommand{\name}[1]{\textbf{\Large #1}}
\newcommand{\separator}{$|$}

\begin{document}

%-------ENCABEZADO-------
\begin{center}
  \name{Christian Bueno} \\[4pt]
  \small
  \href{mailto:chris@deployhero.dev}{\uline{chris@deployhero.dev}} \separator{}
  +593 99 028 8710 \separator{}
  \href{https://www.christianbueno.work}{\uline{christianbueno.work}} \separator{}
  \href{https://www.linkedin.com/in/christianbueno1/}{\uline{LinkedIn}} \separator{}
  \href{https://github.com/christianbueno1}{\uline{GitHub}}
\end{center}

\vspace{12pt}

%-------DESTINATARIO-------
\noindent
Laura Zambrano \\
Gerente de Recursos Humanos \\
Banco de Guayaquil \\
Pichincha 107 y P. Icaza \\
Guayaquil, Ecuador

\vspace{12pt}

%-------FECHA-------
\noindent
Guayaquil, \today

\vspace{12pt}

%-------SALUDO-------
\noindent
Estimada Laura Zambrano:

\vspace{8pt}

%-------CUERPO DE LA CARTA-------
\noindent
Me dirijo a ustedes con gran interés en formar parte del equipo de \textbf{Banco de Guayaquil}. Como Desarrollador Full Stack y Especialista DevOps con más de 6 años de experiencia, estoy seguro de que puedo aportar valor significativo a su institución.

\vspace{8pt}

\noindent
Mi experiencia incluye el desarrollo de aplicaciones web modernas utilizando tecnologías como \textbf{React}, \textbf{TypeScript}, \textbf{FastAPI} y \textbf{Django}, así como la implementación de infraestructura automatizada con \textbf{Podman}, \textbf{OpenTofu} y \textbf{Ansible}. Durante mi trayectoria, he desplegado más de 7 proyectos en producción, incluyendo plataformas e-commerce, sistemas SaaS y aplicaciones corporativas, trabajando con proveedores cloud como DigitalOcean, Azure y AWS.

\vspace{8pt}

\noindent
En mi rol más reciente en Nostrum, automaticé el procesamiento de datos con Python y CustomTkinter, logrando reducir el tiempo de procesamiento en un \textbf{75\%}. Esta experiencia demuestra mi capacidad para identificar oportunidades de mejora y desarrollar soluciones eficientes que generan impacto medible.

\vspace{8pt}

\noindent
Actualmente estoy cursando el último año de Ingeniería en Computación en ESPOL, donde he complementado mi formación práctica con fundamentos sólidos en ciencias de la computación. Busco oportunidades que me permitan combinar desarrollo full stack con prácticas DevOps modernas, contribuyendo a proyectos desafiantes que requieran tanto habilidades técnicas como pensamiento estratégico.

\vspace{8pt}

\noindent
Me entusiasma la posibilidad de contribuir a \textbf{Banco de Guayaquil} y estaría encantado de discutir cómo mi experiencia se alinea con sus necesidades. Quedo a su disposición para una entrevista en la que podamos conversar más sobre esta oportunidad.

\vspace{8pt}

%-------DESPEDIDA-------
\noindent
Agradezco de antemano su tiempo y consideración.

\vspace{8pt}

\noindent
Atentamente,

\vspace{16pt}

\noindent
\textbf{Christian Bueno}

\end{document}
