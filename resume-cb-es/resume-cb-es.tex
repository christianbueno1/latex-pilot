%-------------------------
% Hoja de Vida, una plantilla de currículum en latex para desarrolladores
% Autor : Christian Bueno
% Basado en: https://github.com/christianbueno1/latex-pilot
% Licencia : MIT
%-------------------------



%------------PAQUETES----------------
\documentclass[a4paper,11pt]{article}

% Paquetes de diseño y tipografía
\usepackage{titlesec}     % formato personalizado de secciones
\usepackage{enumitem}     % control sobre listas (itemize, enumerate)
\usepackage{fancyhdr}     % personalización de encabezados y pies de página
\usepackage[empty]{fullpage} % márgenes de página personalizados

% Enlaces y formato de texto
\usepackage[hidelinks]{hyperref} % enlaces clicables sin bordes
\usepackage[normalem]{ulem}      % subrayado con soporte para saltos de línea

% Soporte de color (versión moderna)
\usepackage{xcolor}       % gestión avanzada de color

% Soporte de idioma
\usepackage[spanish,es-noquoting]{babel} % reglas tipográficas del español
%-----------------------------------------

\input glyphtounicode % convierte nombres de glifos a unicode
\pdfgentounicode=1 % asegura que los PDFs generados sean legibles por ATS

%----------OPCIONES DE FUENTE-------------------
\usepackage[default]{sourcesanspro} % usa la fuente Source Sans Pro
\urlstyle{same} % los URLs usan la misma fuente que el documento
%-----------------------------------------


%----------OPCIONES DE MÁRGENES-----------------
\pagestyle{fancy} % estilo de página personalizable con fancyhdr
\fancyhf{} % limpia todos los campos de encabezado y pie de página

\renewcommand{\headrulewidth}{0in} % grosor de línea bajo el encabezado a cero
\renewcommand{\footrulewidth}{0in} % grosor de línea sobre el pie de página a cero

\setlength{\tabcolsep}{0in} % grosor del separador de columnas en tablas a cero
\setlength{\footskip}{4.08003pt} % espacio mínimo para pie de página (evita warnings)

% origen del documento es una pulgada desde arriba y desde la izquierda
% oddsidemargin y evensidemargin se refieren al margen izquierdo
% el margen derecho se ajusta indirectamente usando oddsidemargin
\addtolength{\oddsidemargin}{-0.5in}
\addtolength{\topmargin}{-0.5in}

\addtolength{\textwidth}{1.0in} % aumenta el ancho del área de texto en una pulgada
\addtolength{\textheight}{1.0in} % aumenta la altura del área de texto en una pulgada

\raggedbottom{} % todas las páginas tienen la altura de la página actual, sin espacio vertical extra
\raggedright{} % todas las páginas tienen el ancho de la página actual, sin espacio horizontal extra
%------------------------------------------


%--------COMANDOS DE SECCIÓN---------
% \titleformat{<comando>}
%   [<forma>]{<formato>}{<etiqueta>}{<separación>}
%     {<código-antes>}[<código-después>]

% comando: el comando de sección a redefinir (ej: \section)
% forma: estilo de la fuente; scshape = versalitas (small caps)
% formato: formato aplicado al título completo - etiqueta y texto
% etiqueta: define la etiqueta de la sección
% separación: separación horizontal entre etiqueta y cuerpo del título
% código-antes: código ejecutado antes del título
% código-después: código ejecutado después del título

\titleformat{\section}
  {\scshape\large}{}
    {0em}{\color{blue}}[\color{black}\titlerule\vspace{0pt}]
%-------------------------------------


%--------REDEFINICIONES----------------
% redefine la profundidad del subrayado a 2pt
\renewcommand{\ULdepth}{2pt}
%-------------------------------------


%--------COMANDOS PERSONALIZADOS--------------
% \vspace{} define un espacio vertical de tamaño dado, modificarlo en comandos personalizados 
% ayuda a estirar o encoger el CV para eliminar o agregar contenido

% resumeItem renderiza un punto de lista
\newcommand{\resumeItem}[1]{
  \item\small{#1}
}

% comandos para iniciar y finalizar itemización de resumeItem
% rightmargin ajustado a 0.11in para evitar desbordamiento de resumeItem
\newcommand{\resumeItemListStart}{\begin{itemize}[rightmargin=0.11in]}
\newcommand{\resumeItemListEnd}{\end{itemize}}

% resumeQuadHeading renderiza cuatro elementos en dos columnas
% segunda fila en cursiva y primer elemento de primera fila en negrita
% se puede usar para experiencia y proyectos con cuatro elementos
\newcommand{\resumeQuadHeading}[4]{
  \item
  \begin{tabular*}{0.96\textwidth}[t]{l@{\extracolsep{\fill}}r}
    \textbf{#1} & #2 \\
    \textit{\small#3} & \textit{\small #4} \\
  \end{tabular*}
}

% comandos para iniciar y finalizar itemización de resumeQuadHeading
% lefmargin para indentación izquierda de 0.15in para resumeItems
\newcommand{\resumeHeadingListStart}{
  \begin{itemize}[leftmargin=0.15in, label={}]
}
\newcommand{\resumeHeadingListEnd}{\end{itemize}}

% comandos para detalles de contacto
\newcommand{\name}[1]{\textbf{\Huge #1}}
\newcommand{\socialLinks}[1]{#1}
\newcommand{\separator}{$|$}
%-------------------------------------------


%__________________HOJA DE VIDA____________________
% Puedes reorganizar las secciones en el orden que prefieras
\begin{document}

%-----------DETALLES DE CONTACTO------------------

\begin{tabular*}{\textwidth}{l@{\extracolsep{\fill}}r}
  \name{Christian Bueno \vspace{2pt}} & 
  Ubicación: Guayaquil, Guayas, Ecuador \\
  
  \socialLinks{
    \href{https://www.christianbueno.work}{\uline{christianbueno.work}} \separator{}
    \href{https://www.linkedin.com/in/christianbueno1/}{\uline{LinkedIn}} \separator{}
    \href{https://github.com/christianbueno1}{\uline{GitHub}} \separator{}
    \href{https://www.instagram.com/christianbueno.1/}{\uline{Instagram}}
  } & 
  Email: \href{mailto:chris@deployhero.dev}{\uline{chris@deployhero.dev}} \\
  
  & móvil: +593 99 028 8710 \\
\end{tabular*}
%--------------------------------------------


%--------------HABILIDADES TÉCNICAS------------------------
\section{Habilidades Técnicas}
\begin{tabular}{@{}p{3.2cm}l@{}}
  \textbf{DevOps:} & Jenkins, Terraform/OpenTofu, Ansible, Kubernetes \\
  \textbf{Lenguajes:} & Python, JavaScript, TypeScript, Java \\
  \textbf{Frameworks:} & NestJS, Express.js, FastAPI, Django, Flask, Tailwindcss, Next.js \\
  \textbf{Bases de Datos:} & MySQL/MariaDB, PostgreSQL, SQL Server, MongoDB, Apache Cassandra \\
  \textbf{Herramientas:} & Git, Vite, AWS, Linode, Docker, Podman, Linux, Node.js \\
  \textbf{Diseño:} & Kdenlive, Inkscape, GIMP, Canva, ChatGPT, Midjourney, OpenAI API \\
\end{tabular}
%--------------------------------------------


%-----------EXPERIENCIA-----------------------
% Destila todos tus puntos clave en viñetas pequeñas siguiendo el patrón "desafío-acción-resultado"
% para máxima eficiencia. Cuantifica (usa números) tus logros cuando sea posible, resalta palabras importantes

\section{Experiencia}
\resumeHeadingListStart{}

\resumeQuadHeading{Analista de Datos}{Ag 2024 -- Set 2024}
  {\href{https://www.guayaquil.gob.ec/}{\uline{Nostrum}}}{Guayaquil, Guayas, Ecuador}
  \resumeItemListStart{}
    \resumeItem{Automaticé coincidencia de direcciones con \textbf{Python} y \textbf{CustomTkinter}, reduciendo tiempo de procesamiento en \textbf{75\%}.}
    \resumeItem{Procesé múltiples archivos Excel con algoritmo eficiente usando \textbf{Pandas} y umbrales de similitud configurables.}
    \resumeItem{Implementé arquitectura modular separando UI, lógica de negocio y procesamiento de datos para mejorar mantenibilidad.}
    \resumeItem{Integré seguimiento de progreso en tiempo real para datasets con miles de registros.}
\resumeItemListEnd{}

\resumeQuadHeading{Desarrollo Web}{Jul 2023 -- Ene 2024}
  {\href{https://contabilly.com}{\uline{Contabilly S.A.S.}}}{Guayaquil, Guayas, Ecuador}
    \resumeItemListStart{}
      \resumeItem{Desarrollé sistema de gestión de ventas con agendamiento de citas y gestión de contactos.}
      \resumeItem{Construí frontend con \textbf{React}, \textbf{Redux Toolkit}, validación con \textbf{Formik/Yup} y \textbf{Material UI}.}
      \resumeItem{Implementé backend con \textbf{Express.js}, \textbf{MySQL} y \textbf{Sequelize} como ORM.}
    \resumeItemListEnd{}

  \resumeQuadHeading{Desarrollo Web}{Oct 2022 -- Dic 2022}
  {\href{https://high-track.com/}{\uline{High-Track}}}{Guayaquil, Guayas, Ecuador}
    \resumeItemListStart{}
      \resumeItem{Implementé sistema de validación para gestión de estudiantes, buses, conductores y rutas escolares.}
      \resumeItem{Desarrollé módulo de rastreo GPS-NMEA usando estándar \textbf{RMC/\$GPRMC} para tracking de vehículos en tiempo real.}
      \resumeItem{Trabajé con \textbf{Django}, \textbf{Bootstrap}, \textbf{jQuery} mejorando la experiencia de usuario en \textbf{10\%}.}
    \resumeItemListEnd{}
    
  \resumeQuadHeading{Desarrollador Full Stack}{Ene 2018 -- Presente}
  {\href{https://christianbueno.work}{\uline{www.christianbueno.work}}}{Guayaquil, Guayas, Ecuador}
    \resumeItemListStart{}
      \resumeItem{Desarrollé y desplegué \textbf{7 proyectos} (e-commerce, corporativos, herramientas SaaS) usando \textbf{React}, \textbf{TypeScript}, \textbf{FastAPI}, \textbf{WordPress} y \textbf{Tailwind CSS}.}
      \resumeItem{Automaticé despliegues con \textbf{Podman}, \textbf{Shell scripts}, construyendo imágenes, pods, contenedores y volúmenes para persistencia de datos.}
      \resumeItem{Gestioné infraestructura como código con \textbf{OpenTofu} (DigitalOcean, Azure, AWS) y \textbf{Ansible} para provisionamiento de VMs, configuración de firewalls y servidores web en \textbf{RedHat/Rocky/Ubuntu Linux}.}
      \resumeItem{Administré DNS con \textbf{Cloudflare} y entregué proyectos completos incluyendo e-commerce, landing pages, apps móviles y sistemas corporativos.}
    \resumeItemListEnd{}
    
\resumeHeadingListEnd{}
%---------------------------------------------




%-----------RESUMEN--------------------------
% Mantén esto corto, simple y directo al punto

\section{Biografía}
\small{
  \textbf{Desarrollador Full Stack y Especialista DevOps} con más de 6 años de experiencia construyendo y desplegando soluciones tecnológicas. Estudiante de Ingeniería en Computación en ESPOL (próximo a graduarse). Me especializo en desarrollo de aplicaciones web modernas, automatización de infraestructura y despliegues en producción. Trabajo con tecnologías como \textbf{React}, \textbf{TypeScript}, \textbf{FastAPI}, \textbf{Django} y \textbf{Kubernetes}. En DevOps, uso \textbf{Podman}, \textbf{OpenTofu}, \textbf{Ansible} y \textbf{CI/CD} para escalar proyectos de forma confiable. He desplegado 7+ proyectos en producción incluyendo e-commerce, SaaS y sistemas corporativos. Busco oportunidades para trabajar en proyectos desafiantes que combinen desarrollo full stack con prácticas DevOps modernas.
}
%--------------------------------------------


%-----------EDUCACIÓN-------------------------
% Menciona tu promedio académico si es bueno, en la primera fila de la segunda columna

\section{Educación}
  \resumeHeadingListStart{}
    \resumeQuadHeading{Escuela Superior Politécnica del Litoral - ESPOL}{2005 -- Presente}
    {Ingeniería en Computación (en curso)}{Guayaquil, Ecuador}
  \resumeHeadingListEnd{}
%---------------------------------------------

\end{document}
