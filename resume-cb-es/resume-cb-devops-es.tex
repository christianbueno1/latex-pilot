%-------------------------
% Rezume, a latex resume template for developers
% Author : Nanu Panchamurthy
% Based off of: https://github.com/sb2nov/resume
% License : MIT

% Hope this resume template helps you land an awesome job. If you found this helpful, please consider starring the github repo here, .
%-------------------------



%------------PACKAGES----------------
\documentclass[a4paper,11pt]{article}

\usepackage{verbatim} % reimplements the "verbatim" and "verbatim*" environments

\usepackage{titlesec} % provides an interface to sectioning commands i.e. custom elements

\usepackage{color} % provides both foreground and background color management

\usepackage{enumitem} % provides control over enumerate, itemize and description

\usepackage{fancyhdr} % provides extensive facilities for constructing headers, footers and also controlling their use

\usepackage{tabularx} % defines an environment tabularx, extension of "tabular" with an extra designator x, paragraph like column whose width automatically expands to fill the width of the environment

\usepackage{latexsym} % provides mathematical symbols

\usepackage{marvosym} % provides martin vogel's symbol font which contains various symbols

\usepackage[empty]{fullpage} % sets margins to one inch and removes headers, footers etc..

\usepackage[hidelinks]{hyperref} % removes color and shadow of hyperlinks

\usepackage[normalem]{ulem} % provides "\ul" (uline) command which will break at line breaks

\usepackage[english]{babel} % provides culturally determined typographical rules for wide range of languages
%-----------------------------------------

\input glyphtounicode % converts glyph names to unicode
\pdfgentounicode=1 % ensures pdfs generated are ats readable

%----------FONT OPTIONS-------------------
\usepackage[default]{sourcesanspro} % uses the font source sans pro
\urlstyle{same} % changes url font from default urlfont to font being used by the document
%-----------------------------------------


%----------MARGIN OPTIONS-----------------
\pagestyle{fancy} % set page style to one configured by fancyhdr
\fancyhf{} % clear all header and footer fields

\renewcommand{\headrulewidth}{0in} % sets thickness of linerule under header to zero
\renewcommand{\footrulewidth}{0in} % sets thickness of linerule over footer to zero

\setlength{\tabcolsep}{0in} % sets thickness of column separator in tables to zero

% origin of the document is one inch from the top and from and the left
% oddsidemargin and evensidemargin both refer to the left margin
% right margin is indirectly set using oddsidemargin
\addtolength{\oddsidemargin}{-0.5in}
\addtolength{\topmargin}{-0.5in}

\addtolength{\textwidth}{1.0in} % sets width of text area in the page to one inch
\addtolength{\textheight}{1.0in} % sets height of text area in the page to one inch

\raggedbottom{} % makes all pages the height of current page, no extra vertical space added
\raggedright{} % makes all pages the width of current page, no extra horizontal space added
%------------------------------------------


%--------SECTIONING COMMANDS---------
% \titleformat{<command>}
%   [<shape>]{<format>}{<label>}{<sep>}
%     {<before-code>}[<after-code>]

% command is the sectioning command to be redefined
% shape is the style of the font; scshape stands for small caps style
% format is the format to be applied to whole title- label and text; absent here
% label defines the label
% sep is the horizontal separation between label and title body
% before-code is the code to be executed before
% after-code is the code to be executed after

\titleformat{\section}
  {\scshape\large}{}
    {0em}{\color{blue}}[\color{black}\titlerule\vspace{0pt}]
%-------------------------------------


%--------REDEFINITIONS----------------
% redefines the style of the bullet point
\renewcommand\labelitemii{$\vcenter{\hbox{\tiny$\bullet$}}$}

% redefines the underline depth to 2pt
\renewcommand{\ULdepth}{2pt}
%-------------------------------------


%--------CUSTOM COMMANDS--------------
%\vspace{} defines a vertical space of given size, modifying this in custom commands can help stretch or shrink resume to remove or add content

% resumeItem renders a bullet point
\newcommand{\resumeItem}[1]{
  \item\small{#1}
}

% commands to start and end itemization of resumeItem, rightmargin set to 0.11in to avoid the overflow of resumetItem beyond whatever resumeItemHeading is being used
\newcommand{\resumeItemListStart}{\begin{itemize}[rightmargin=0.11in]}
\newcommand{\resumeItemListEnd}{\end{itemize}}

% resumeSectionType renders a bolded type to be used under a section, used as skill type here, middle element is used to keep ":"s in the same vertical line
\newcommand{\resumeSectionType}[3]{
  % \item\begin{tabular*}{0.96\textwidth}[t]{
  \item\begin{tabularx}{0.96\textwidth}[t]{
    % p{0.15\linewidth}p{0.02\linewidth}p{0.81\linewidth}
    p{0.15\linewidth}p{0.02\linewidth}X
  }
    \textbf{#1} & #2 & #3
  % \end{tabular*}\vspace{-2pt}
  \end{tabularx}\vspace{-2pt}
}

% resumeTrioHeading renders three elements in three columns with second element being italicized and first element bolded, can be used for projects with three elements
\newcommand{\resumeTrioHeading}[3]{
  \item\small{
    \begin{tabular*}{0.96\textwidth}[t]{
      l@{\extracolsep{\fill}}c@{\extracolsep{\fill}}r
    }
      \textbf{#1} & \textit{#2} & #3
    \end{tabular*}
  }
}

% resumeQuadHeading renders four elements in a two columns with the second row being italicized and first element of first row bolded, can be used for experience and projects with four elements
\newcommand{\resumeQuadHeading}[4]{
  \item
  \begin{tabular*}{0.96\textwidth}[t]{l@{\extracolsep{\fill}}r}
    \textbf{#1} & #2 \\
    \textit{\small#3} & \textit{\small #4} \\
  \end{tabular*}
}

% resumeQuadHeadingChild renders the second row of resumeQuadHeading, can be used for experience if different roles in the same company need to added
\newcommand{\resumeQuadHeadingChild}[2]{
  \item
  \begin{tabular*}{0.96\textwidth}[t]{l@{\extracolsep{\fill}}r}
    \textbf{\small#1} & {\small#2} \\
  \end{tabular*}
}

% commands to start and end itemization of resumeQuadHeading, lefmargin for left indent of 0.15in for resumeItems
\newcommand{\resumeHeadingListStart}{
  \begin{itemize}[leftmargin=0.15in, label={}]
}
\newcommand{\resumeHeadingListEnd}{\end{itemize}}
%-------------------------------------------


%__________________RESUME____________________
% You can rearrange sections in any order you may prefer
\begin{document}

%-----------CONTACT DETAILS------------------
% Make sure all the details are correct, you can add more links in the first row of second column if needed

\begin{tabular*}{\textwidth}{l@{\extracolsep{\fill}}r}
  \textbf{\Huge Christian Bueno \vspace{2pt}} & % row = 1, col = 1
  Location: Guayaquil, Guayas, Ecuador \\ % row = 1, col = 2
  
  \href{https://www.christianbueno.work}{\uline{christianbueno.work}} $|$ % row = 2, col = 1
  \href{https://www.linkedin.com/in/christianbueno1/}{\uline{LinkedIn}} $|$ % row = 2, col = 1
  \href{https://github.com/christianbueno1}{\uline{GitHub}} $|$ % row = 2, col = 1
  \href{https://www.instagram.com/christianbueno.1/}{\uline{Instagram}} $|$ % row = 2, col = 1
  \href{https://codepen.io/christianbueno1}{\uline{CodePen}} & % row = 2, col = 1
  Email: \href{mailto:chris@deployhero.dev}{\uline{chris@deployhero.dev}} \\ % row = 2, col = 2
  
   & % row = 3, col = 1
  Mobile: +593 99 028 8710 \\ % row = 3, col = 2
\end{tabular*}
%--------------------------------------------


%--------------SKILLS------------------------
% Add or remove resumeSectionTypes according to your needs

\section{Technical Skills}
  \resumeHeadingListStart{}
  
    \resumeSectionType{DevOps}{:}{Jenkins, Terraform, Ansible, Kubernetes}
    
    \resumeSectionType{Languages}{:}{Python, JavaScript, TypeScript, Java}
    
    \resumeSectionType{Frameworks}{:}{NestJS, Express.js, FastAPI, Django, Flask, Tailwindcss, Next.js}
    
    \resumeSectionType{Databases}{:}{MySQL/MariaDB, PostgreSQL, SQL Server, MongoDB, Apache Cassandra}
    
    \resumeSectionType{Dev Tools}{:}{Git, Vite, AWS, Linode, Docker, Podman, Linux, Node.js}
    
    \resumeSectionType{Design}{:}{Kdenlive, Inkscape, GIMP, Canva, ChatGPT, Midjourney, OpenAI API, GitHub Copilot}

  \resumeHeadingListEnd{}
%--------------------------------------------


%-----------EXPERIENCE-----------------------
% Distill all your talking points to small bullet points which follow the pattern "challenge-action-result" for maximum efficiency. Try to quantify (use numbers) your points whenver possible, highlight words of importance

\section{Experiencia}
\resumeHeadingListStart{}

\resumeQuadHeading{Analista de Datos}{Ag 2024 -- Set 2024}
  {\href{https://www.guayaquil.gob.ec/}{\uline{Nostrum}}}{Guayaquil, Guayas, Ecuador}
  \small {Contacto: Ing. Michelle Prendes, +593 99 927 9601}
  \resumeItemListStart{}
    \resumeItem{Desarrollé una aplicación GUI basada en Python utilizando CustomTkinter para la automatización de la coincidencia de direcciones y el procesamiento de datos.}
    \resumeItem{Implementé un algoritmo eficiente de coincidencia de direcciones para procesar múltiples hojas de Excel con umbrales de similitud configurables.}
    \resumeItem{Creé una arquitectura modular para separar la interfaz de usuario, el procesamiento de datos y la lógica de coincidencia de direcciones, mejorando así la mantenibilidad.}
    \resumeItem{Integré una funcionalidad de seguimiento del progreso para conjuntos de datos grandes, proporcionando retroalimentación en tiempo real al usuario.}
    \resumeItem{Automaticé operaciones con archivos de Excel, reduciendo el tiempo de entrada manual de datos en aproximadamente un 75\%.}
    \resumeItem{Utilicé tecnologías como Python, Pandas y CustomTkinter para un manejo efectivo de datos y una interacción fluida con el usuario.}
\resumeItemListEnd{}

\resumeQuadHeading{Desarrollo Web}{Jul 2023 -- Ene 2024}
  {\href{https://contabilly.com}{\uline{Contabilly S.A.S.}}}{Guayaquil, Guayas, Ecuador}
  \small {Contactos: Ing. Leonardo Castro, +593 96 270 3372 | Ing. Leo Castro Jr, +593 99 257 1921}
    \resumeItemListStart{}
      \resumeItem{Trabajé en un software de gestión de ventas con funciones como gestionar contactos telefónicos y programar citas con clientes potenciales.}
      \resumeItem{En frontend utilicé React, Redux Toolkit, validación con Formik, Yup y Material UI.}
      \resumeItem{En backend usé Express.js, MySQL y Sequelize.}
    \resumeItemListEnd{}

  \resumeQuadHeading{Desarrollo Web}{Oct 2022 -- Dic 2022}
  {\href{https://high-track.com/}{\uline{High-Track}}}{Guayaquil, Guayas, Ecuador}
  \small {Contactos: Ing. Milton Villafuerte, +593 99 943 7032 | Ing. Jonathan Bravo, +593 99 066 0821}
    \resumeItemListStart{}
      \resumeItem{Implementé la validación para controlar estudiantes, buses, conductores, escuelas y rutas en una aplicación de rastreo de buses.}
      \resumeItem{Diseñé la opción "Rutas de buses" e implementé una frase GPS-NMEA usando el estándar RMC o \$GPRMC para rastrear vehículos.}
      \resumeItem{Trabajé con Bootstrap, Django, jQuery y Git.}
      \resumeItem{Mejoré la utilidad de la aplicación en un 10\%.}
    \resumeItemListEnd{}
    
  \resumeQuadHeading{Desarrollador Full Stack}{Ene 2018 -- Presente}
  {\href{https://christianbueno.work}{\uline{www.christianbueno.work}}}{Guayaquil, Guayas, Ecuador}
    \resumeItemListStart{}
      \resumeItem{Servicios de Diseño y Desarrollo Web.}
      \resumeItem{Servicios de Desarrollo de Software.}
      \resumeItem{Desarrollo de Aplicaciones Móviles.}
      \resumeItem{Creación de landing pages, sitios corporativos, e-commerce y tiendas online.}
      \resumeItem{Creación de contenido, diseño gráfico y multimedia, gestión de redes sociales (community manager).}
    \resumeItemListEnd{}
    
\resumeHeadingListEnd{}
%---------------------------------------------




%-----------SUMMARY--------------------------
% Keep this short, simple and straigth to point

\section{Biografía}
\small{
  ¡Hola! Soy Christian Bueno, \textbf{Desarrollador Full Stack y Especialista DevOps} con más de 6 años de experiencia creando soluciones tecnológicas desde Guayaquil, Ecuador. Me gradué en Ciencias de la Computación en ESPOL y mi pasión por el código empezó desde la escuela. Me especializo en construir aplicaciones web con interfaces atractivas, sistemas backend eficientes y despliegues automatizados. Trabajo con tecnologías como Next.js, Tailwind, FastAPI, Django y Kubernetes. Diseño soluciones que combinan rendimiento, accesibilidad y experiencia de usuario. En el área DevOps, uso herramientas como Terraform, Jenkins y Podman para escalar proyectos de forma confiable. Mi enfoque siempre es desarrollar con propósito, automatizar lo repetitivo y mantener el código limpio y documentado.
}
%--------------------------------------------


%-----------EDUCATION-------------------------
% Mention your CGPA, if its good, in the first row of second column

\section{Educación}
  \resumeHeadingListStart{}
    \resumeQuadHeading{Escuela Superior Politécnica del Litoral - ESPOL}{Guayaquil, Guayas, Ecuador}
    {Ingeniería en Computación}{}
  \resumeHeadingListEnd{}
%---------------------------------------------

\end{document}
