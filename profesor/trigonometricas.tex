\documentclass{profesor}

\title{Razones Trigonométricas - Ejercicios y Resoluciones}

\begin{document}

\maketitle
\tableofcontents
\newpage

\section{Razones Trigonométricas}

% --- VALORES DE LAS RAZONES TRIGONOMÉTRICAS DE ÁNGULOS NOTABLES ---
\subsection{Valores de las Razones Trigonométricas de Ángulos Notables}

\textbf{Ejercicio 1:}  
Determinar los valores exactos de las razones trigonométricas para \( 30^\circ \).

\textbf{Solución:}  
\begin{itemize}
    \item Usando la tabla de valores notables:
    \[
    \sin 30^\circ = \frac{1}{2}, \quad \cos 30^\circ = \frac{\sqrt{3}}{2}, \quad \tan 30^\circ = \frac{1}{\sqrt{3}}
    \]
    \item Además:
    \[
    \csc 30^\circ = 2, \quad \sec 30^\circ = \frac{2}{\sqrt{3}}, \quad \cot 30^\circ = \sqrt{3}
    \]
\end{itemize}

\textbf{Ejercicio 2:}  
Calcular \( \tan 45^\circ \) y \( \cot 45^\circ \).

\textbf{Solución:}  
\begin{itemize}
    \item De la tabla de razones notables:
    \[
    \tan 45^\circ = \frac{\sin 45^\circ}{\cos 45^\circ} = \frac{\frac{\sqrt{2}}{2}}{\frac{\sqrt{2}}{2}} = 1
    \]
    \item Y la cotangente es su inversa:
    \[
    \cot 45^\circ = \frac{1}{\tan 45^\circ} = 1
    \]
\end{itemize}

\newpage
% --- RESOLUCIÓN DE TRIÁNGULOS ---
\subsection{Resolución de Triángulos}

\textbf{Ejercicio 1:}  
En un triángulo rectángulo, si \( \theta = 60^\circ \) y el cateto opuesto mide 10 cm, hallar la hipotenusa.

\textbf{Solución:}  
\begin{itemize}
    \item Se usa la función seno:
    \[
    \sin 60^\circ = \frac{\text{cateto opuesto}}{\text{hipotenusa}}
    \]
    \item Sustituyendo valores:
    \[
    \frac{\sqrt{3}}{2} = \frac{10}{h}
    \]
    \item Despejamos \( h \):
    \[
    h = \frac{10 \times 2}{\sqrt{3}} = \frac{20}{\sqrt{3}} \approx 11.55 \text{ cm}
    \]
\end{itemize}

\textbf{Ejercicio 2:}  
En un triángulo rectángulo, el cateto adyacente mide 5 cm y el ángulo \( \theta = 45^\circ \). Hallar la hipotenusa.

\textbf{Solución:}  
\begin{itemize}
    \item Se usa la función coseno:
    \[
    \cos 45^\circ = \frac{\text{cateto adyacente}}{\text{hipotenusa}}
    \]
    \item Sustituyendo valores:
    \[
    \frac{\sqrt{2}}{2} = \frac{5}{h}
    \]
    \item Despejamos \( h \):
    \[
    h = \frac{5 \times 2}{\sqrt{2}} = 5\sqrt{2} \approx 7.07 \text{ cm}
    \]
\end{itemize}

\newpage
% --- PROBLEMAS DE APLICACIÓN ---
\subsection{Problemas de Aplicación}

\textbf{Ejercicio 1:}  
Un avión vuela a una altitud de 2 km y observa un punto en el suelo con un ángulo de depresión de \( 30^\circ \). ¿A qué distancia horizontal está el punto observado?

\textbf{Solución:}  
\begin{itemize}
    \item Se forma un triángulo rectángulo donde la altura es 2 km y el ángulo de depresión es \( 30^\circ \).
    \item Se usa la función tangente:
    \[
    \tan 30^\circ = \frac{\text{altura}}{\text{distancia horizontal}}
    \]
    \item Sustituyendo valores:
    \[
    \frac{1}{\sqrt{3}} = \frac{2}{x}
    \]
    \item Despejamos \( x \):
    \[
    x = 2\sqrt{3} \approx 3.46 \text{ km}
    \]
\end{itemize}

\textbf{Ejercicio 2:}  
Un poste de 8 m de altura proyecta una sombra de 6 m. ¿Cuál es el ángulo de elevación del sol?

\textbf{Solución:}  
\begin{itemize}
    \item Se usa la función tangente:
    \[
    \tan \theta = \frac{\text{altura}}{\text{sombra}}
    \]
    \item Sustituyendo valores:
    \[
    \tan \theta = \frac{8}{6} = \frac{4}{3}
    \]
    \item Aplicando arco tangente:
    \[
    \theta = \tan^{-1} \left( \frac{4}{3} \right) \approx 53.13^\circ
    \]
\end{itemize}

\end{document}
