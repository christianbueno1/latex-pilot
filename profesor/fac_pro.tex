\documentclass{profesor}

\title{Procedimiento de Resolución - Ejercicios de Factorización}

\begin{document}

\maketitle
\tableofcontents
\newpage

\section{Factorización - Procedimientos Detallados}

\subsection{Factor común monomio}
\textbf{Ejercicio 1:}  
Factorizar \( 6x^3 + 9x^2 \).

\textbf{Solución:}
\begin{itemize}
    \item Se identifica el factor común: \( 3x^2 \).
    \item Se extrae el factor común:
    \[
    6x^3 + 9x^2 = 3x^2 (2x + 3)
    \]
\end{itemize}

\textbf{Ejercicio 5:}  
Factorizar \( 12p^5q^2 - 18p^3q^4 + 24p^2q \).

\textbf{Solución:}
\begin{itemize}
    \item Factor común: \( 6p^2q \).
    \item Se extrae el factor común:
    \[
    12p^5q^2 - 18p^3q^4 + 24p^2q = 6p^2q (2p^3q - 3p q^3 + 4)
    \]
\end{itemize}

\subsection{Diferencia de cuadrados}
\textbf{Ejercicio 1:}  
Factorizar \( x^2 - 16 \).

\textbf{Solución:}
\begin{itemize}
    \item Se escribe como diferencia de cuadrados:
    \[
    x^2 - 16 = (x - 4)(x + 4)
    \]
\end{itemize}

\textbf{Ejercicio 5:}  
Factorizar \( x^4 - 81 \).

\textbf{Solución:}
\begin{itemize}
    \item Se reconoce como \( (x^2)^2 - 9^2 \), aplicando la diferencia de cuadrados:
    \[
    x^4 - 81 = (x^2 - 9)(x^2 + 9)
    \]
    \item Luego, se sigue factorizando \( x^2 - 9 \):
    \[
    x^4 - 81 = (x - 3)(x + 3)(x^2 + 9)
    \]
\end{itemize}

\subsection{Suma o diferencia de cubos}
\textbf{Ejercicio 1:}  
Factorizar \( x^3 + 27 \).

\textbf{Solución:}
\begin{itemize}
    \item Se usa la fórmula de la suma de cubos:
    \[
    a^3 + b^3 = (a + b)(a^2 - ab + b^2)
    \]
    \item Identificamos \( a = x \) y \( b = 3 \):
    \[
    x^3 + 27 = (x + 3)(x^2 - 3x + 9)
    \]
\end{itemize}

\textbf{Ejercicio 5:}  
Factorizar \( 343p^3 + 512q^3 \).

\textbf{Solución:}
\begin{itemize}
    \item Se identifican \( a = 7p \) y \( b = 8q \).
    \item Aplicamos la fórmula:
    \[
    343p^3 + 512q^3 = (7p + 8q)(49p^2 - 56pq + 64q^2)
    \]
\end{itemize}

\subsection{Suma o diferencia de potencias impares iguales}
\textbf{Ejercicio 1:}  
Factorizar \( x^5 - y^5 \).

\textbf{Solución:}
\begin{itemize}
    \item Se usa la factorización:
    \[
    x^5 - y^5 = (x - y)(x^4 + x^3y + x^2y^2 + xy^3 + y^4)
    \]
\end{itemize}

\textbf{Ejercicio 5:}  
Factorizar \( t^{15} - u^{15} \).

\textbf{Solución:}
\begin{itemize}
    \item Se expresa como \( (t^5)^3 - (u^5)^3 \).
    \item Aplicamos la diferencia de cubos y luego de potencias impares:
    \[
    t^{15} - u^{15} = (t^5 - u^5)(t^{10} + t^5u^5 + u^{10})
    \]
    \item Finalmente, factorizamos \( t^5 - u^5 \) usando la regla anterior.
\end{itemize}

\subsection{Trinomio cuadrado perfecto}
\textbf{Ejercicio 1:}  
Factorizar \( x^2 + 6x + 9 \).

\textbf{Solución:}
\begin{itemize}
    \item Se reconoce como un trinomio cuadrado perfecto:
    \[
    x^2 + 6x + 9 = (x + 3)^2
    \]
\end{itemize}

\textbf{Ejercicio 5:}  
Factorizar \( 25p^2 + 70p + 49 \).

\textbf{Solución:}
\begin{itemize}
    \item Se reconoce que \( 25p^2 = (5p)^2 \) y \( 49 = 7^2 \).
    \item Se tiene:
    \[
    25p^2 + 70p + 49 = (5p + 7)^2
    \]
\end{itemize}

\subsection{Factorización por evaluación y división sintética}
\textbf{Ejercicio 1:}  
Factorizar \( x^3 - 4x^2 - 7x + 10 \), con raíz \( x = 2 \).

\textbf{Solución:}
\begin{itemize}
    \item Se realiza división sintética de \( x^3 - 4x^2 - 7x + 10 \) entre \( x - 2 \).
    \[
    \begin{array}{r|rrrr}
    2 & 1 & -4 & -7 & 10  \\
      &   & 2  & -4 & -22 \\
    \hline
      & 1 & -2  & -11 & 0 \\
    \end{array}
    \]
    \item El cociente es \( x^2 - 2x - 5 \).
    \item Se factoriza si es posible.
\end{itemize}

\textbf{Ejercicio 5:}  
Factorizar \( x^4 - 10x^3 + 35x^2 - 50x + 24 \), con raíz \( x = 3 \).

\textbf{Solución:}
\begin{itemize}
    \item Se realiza división sintética:
    \[
    \begin{array}{r|rrrrr}
    3 & 1 & -10 & 35 & -50 & 24  \\
      &   & 3  & -21 & 42 & -24 \\
    \hline
      & 1 & -7  & 14  & -8  & 0 \\
    \end{array}
    \]
    \item El cociente \( x^3 - 7x^2 + 14x - 8 \) se sigue factorizando.
\end{itemize}

\end{document}
