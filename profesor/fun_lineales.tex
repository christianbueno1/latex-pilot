\documentclass{profesor}

\title{Funciones Lineales - Ejercicios y Resoluciones}

\begin{document}

\maketitle
\tableofcontents
\newpage

\section{Funciones Lineales}

% --- ECUACIÓN DE LA RECTA ---
\subsection{Ecuación de la Recta}

\textbf{Ejercicio 1:}  
Encuentra la ecuación de la recta que pasa por los puntos \( A(2,3) \) y \( B(5,7) \).

\textbf{Solución:}  
\begin{itemize}
    \item La ecuación de la recta en su forma pendiente-intersección es:
    \[
    y = mx + b
    \]
    \item La pendiente \( m \) se calcula con la fórmula:
    \[
    m = \frac{y_2 - y_1}{x_2 - x_1}
    \]
    \[
    m = \frac{7 - 3}{5 - 2} = \frac{4}{3}
    \]
    \item Usamos el punto \( A(2,3) \) para encontrar \( b \):
    \[
    3 = \frac{4}{3}(2) + b
    \]
    \[
    3 = \frac{8}{3} + b
    \]
    \[
    b = 3 - \frac{8}{3} = \frac{9}{3} - \frac{8}{3} = \frac{1}{3}
    \]
    \item La ecuación de la recta es:
    \[
    y = \frac{4}{3}x + \frac{1}{3}
    \]
\end{itemize}

\textbf{Ejercicio 2:}  
Determina la ecuación de la recta que tiene pendiente \( m = -2 \) y pasa por el punto \( P(3,5) \).

\textbf{Solución:}  
\begin{itemize}
    \item Usamos la ecuación punto-pendiente:
    \[
    y - y_1 = m(x - x_1)
    \]
    \item Sustituyendo los valores:
    \[
    y - 5 = -2(x - 3)
    \]
    \item Expandiendo la ecuación:
    \[
    y - 5 = -2x + 6
    \]
    \[
    y = -2x + 11
    \]
    \item La ecuación de la recta es:
    \[
    y = -2x + 11
    \]
\end{itemize}

\newpage
% --- PROBLEMAS DE APLICACIÓN ---
\subsection{Problemas de Aplicación de las Funciones Lineales}

\textbf{Ejercicio 1:}  
Una compañía de transporte cobra una tarifa base de 5 dólares y 2 dólares por cada kilómetro recorrido. Escribe la ecuación que representa el costo total \( C(x) \) en función de los kilómetros recorridos \( x \) y calcula cuánto se pagará por un viaje de 10 km.

\textbf{Solución:}  
\begin{itemize}
    \item La ecuación del costo es:
    \[
    C(x) = 5 + 2x
    \]
    \item Para \( x = 10 \):
    \[
    C(10) = 5 + 2(10) = 5 + 20 = 25
    \]
    \item El costo del viaje de 10 km es \textbf{25 dólares}.
\end{itemize}

\textbf{Ejercicio 2:}  
Un tanque de agua se llena a razón de 4 litros por minuto. Si inicialmente tiene 20 litros, escribe la función que modela la cantidad de agua en el tanque después de \( t \) minutos y determina cuántos litros habrá después de 15 minutos.

\textbf{Solución:}  
\begin{itemize}
    \item La ecuación es de la forma:
    \[
    A(t) = A_0 + rt
    \]
    donde \( A_0 = 20 \) (agua inicial) y \( r = 4 \) litros por minuto.
    \item La función es:
    \[
    A(t) = 20 + 4t
    \]
    \item Para \( t = 15 \):
    \[
    A(15) = 20 + 4(15) = 20 + 60 = 80
    \]
    \item Habrá \textbf{80 litros} en el tanque después de 15 minutos.
\end{itemize}

\end{document}
