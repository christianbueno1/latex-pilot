\documentclass{profesor}

\title{Semejanza de Triángulos - Ejercicios y Resoluciones}

\begin{document}

\maketitle
\tableofcontents
\newpage

\section{Semejanza de Triángulos}

% --- CRITERIOS DE SEMEJANZA ---
\subsection{Criterios de Semejanza}

\textbf{Ejercicio 1:}  
Determinar si los siguientes triángulos son semejantes. Se tienen los triángulos \( ABC \) y \( DEF \) con las siguientes medidas:  
\[
AB = 6, \quad BC = 8, \quad AC = 10
\]
\[
DE = 9, \quad EF = 12, \quad DF = 15
\]

\textbf{Solución:}  
\begin{itemize}
    \item Calculamos las razones entre los lados correspondientes:
    \[
    \frac{DE}{AB} = \frac{9}{6} = 1.5, \quad \frac{EF}{BC} = \frac{12}{8} = 1.5, \quad \frac{DF}{AC} = \frac{15}{10} = 1.5
    \]
    \item Como las razones son iguales, los triángulos son semejantes por el criterio \textbf{Lado-Lado-Lado (LLL)}.
\end{itemize}

\textbf{Ejercicio 2:}  
En los triángulos \( XYZ \) y \( PQR \), se sabe que los ángulos \( \angle X \) y \( \angle P \) son iguales, y los ángulos \( \angle Y \) y \( \angle Q \) también son iguales. ¿Son semejantes estos triángulos?

\textbf{Solución:}  
\begin{itemize}
    \item Como los triángulos tienen dos ángulos iguales, el tercer ángulo también será igual.
    \item Por lo tanto, los triángulos son semejantes por el criterio \textbf{Ángulo-Ángulo (AA)}.
\end{itemize}

\newpage
% --- PROBLEMAS DE APLICACIÓN ---
\subsection{Problemas de Aplicación}

\textbf{Ejercicio 1:}  
Un poste de luz proyecta una sombra de 5 metros. En el mismo instante, una persona de 1.8 metros de altura proyecta una sombra de 2 metros. ¿Cuál es la altura del poste?

\textbf{Solución:}  
\begin{itemize}
    \item Se forman dos triángulos semejantes entre la persona y el poste con sus sombras.
    \item La proporción es:
    \[
    \frac{\text{altura del poste}}{\text{altura de la persona}} = \frac{\text{longitud de la sombra del poste}}{\text{longitud de la sombra de la persona}}
    \]
    \item Sustituyendo los valores:
    \[
    \frac{x}{1.8} = \frac{5}{2}
    \]
    \item Resolviendo:
    \[
    x = \frac{1.8 \times 5}{2} = 4.5 \text{ metros}
    \]
\end{itemize}

\textbf{Ejercicio 2:}  
Un edificio proyecta una sombra de 15 metros. Un árbol cercano proyecta una sombra de 6 metros. Si la altura del árbol es de 4 metros, ¿cuál es la altura del edificio?

\textbf{Solución:}  
\begin{itemize}
    \item Se aplican triángulos semejantes:
    \[
    \frac{\text{altura del edificio}}{\text{altura del árbol}} = \frac{\text{sombra del edificio}}{\text{sombra del árbol}}
    \]
    \item Sustituyendo los valores:
    \[
    \frac{x}{4} = \frac{15}{6}
    \]
    \item Resolviendo:
    \[
    x = \frac{4 \times 15}{6} = 10 \text{ metros}
    \]
\end{itemize}

\end{document}
